 \documentclass[prd,11pt,letterpaper]{revtex4}
 \usepackage{graphicx}
 \usepackage{epstopdf}
 \usepackage{amsmath,amssymb}
% \usepackage[round]{natbib} %\citet{} for name (00) and \citep{} for (name 00)
% \bibliographystyle{plain}
 
 \textwidth = 6.5 in
 \textheight = 9 in
 \oddsidemargin = -0.0 in
 \evensidemargin = -0.0 in
 \topmargin = 0.0 in
 \headheight = -0.2 in
 \headsep = 0.0 in
 \parskip = 0.05in
 \parindent = 0.2in
 
%\newcommand{\apj}{ApJ}
\newcommand{\apjl}{ApJ}
\newcommand{\apjs}{ApJS}
\newcommand{\aap}{A \& A}
%\newcommand{\aj}{AJ}
\newcommand{\mnras}{MNRAS}
%\newcommand{\physrep}{Physics Reports}
%\newcommand{\prd}{Phys Rev D}
%\newcommand{\nat}{Nature}

\newcommand{\lya}{Ly$\alpha$ }
  \newcommand{\iMpc}{\mbox{ Mpc$^{-1}$}}
   \newcommand{\eV}{\mbox{ eV}}
      \newcommand{\ud}{\mbox{d}}
 
 %%%%%%%%%%%%%%%%%%%%%%%%%%% 
 \begin{document}
 
   \title{Constraining the galaxy luminosity function with Euclid and Hubble}
 
 \author{Jonathan R.~Pritchard}
 \email{j.pritchard@imperial.ac.uk}
\affiliation{Imperial College London}

 %------------------------------------------------------------------------------
 
 %%%%%%%%%%%%%%%%%%%%%%%%%%%%%%%%%%%%%%%%%%%%%%%%%%%%%%%%%%%%%%%%%%%%%%%%%%%%%%%%
 \begin{abstract}

Fisher matrix analysis of the Euclid LBG constraints. How does this interact with Hubble constraints on the faint end for a Schecter function LF.

\end{abstract}
 
 % \keywords{cosmology}
 %------------------------------------------------------------------------------
 % User-supplied List of keywords.
 
% \pacs{98.80.k,98.80.Es}
 
\maketitle

  %%%%%%%%%%%%%%%%%%%%%%%%%%%%%%%%%%%%%%%%%%%%%%%%%%%%%%%%%%%%%%%%%%%%%%%%%%%%%%%%
\section{Introduction} 
\label{sec:intro}


%%%%%%%%%%%%%%%%%%%%%%%%%%%%%%%%%%%%%%%%%%%%%%%%%%%%%%%%%%%%%%%%%%%%%%%%%%%%%%%%
 \bibliography{lgbbib}
%%%%%%%%%%%%%%%%%%%%%%%%%%%%%%%%%%%%%%%%%%%%%%%%%%%%%%%%%%%%%%%%%%%%%%%%%%%%%%%%



%%%%%%%%%%%%%%%%%%%%%%%%%%%%%%%%%%%%%%%%%%%%%%%%%%%%%%%%%%%%%%%%%%%%%%%%%%%%%%%%
 
 \end{document}
 